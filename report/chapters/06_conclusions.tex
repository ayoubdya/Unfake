\chapter{Conclusions and Future Work}
\label{ch:con}

\section{Conclusions}
\label{sec:conclusions}

This project developed UnFake, an AI-powered fake news detection system using machine learning. All objectives were achieved: a RoBERTa-based headline model (74.69\% accuracy, 83.17\% ROC-AUC), a Gradient Boosting article model (99.57\% accuracy), a FastAPI backend, and a user-friendly web interface.

Key findings include: different content types require different approaches; transformers work well for short text while traditional ML excels at document classification; and modern frameworks like FastAPI enable rapid ML API development.

\section{Future Work}
\label{sec:future_work}

Future improvements could include:
\begin{itemize}
    \item Larger transformer models (RoBERTa-large, DeBERTa) for better headline accuracy
    \item Multi-language support for global applicability
    \item Browser extension for seamless user experience
    \item Explainability features using attention visualization or SHAP
    \item Continuous training pipelines to adapt to evolving misinformation tactics
\end{itemize}

The UnFake system demonstrates that machine learning can effectively assist in identifying misinformation, serving as a valuable tool to help users navigate the digital information landscape.