\chapter{Project Setup Guide}
\label{appn:B}

This appendix provides instructions for setting up and running the UnFake system.

\section{Prerequisites}
\label{sec:prerequisites}

\begin{itemize}
    \item Python 3.8 or higher
    \item pip (Python package manager)
    \item CUDA-capable GPU (optional, for faster inference)
    \item Git (for cloning the repository)
\end{itemize}

\section{Installation}
\label{sec:installation}

\subsection{Clone the Repository}
\label{subsec:clone}

\begin{verbatim}
git clone https://github.com/username/unfake.git
cd unfake
\end{verbatim}

\subsection{Create Virtual Environment}
\label{subsec:venv}

\begin{verbatim}
python -m venv venv
source venv/bin/activate  # Linux/macOS
# or
.\venv\Scripts\activate   # Windows
\end{verbatim}

\subsection{Install Dependencies}
\label{subsec:deps}

\begin{verbatim}
pip install -r requirements.txt
\end{verbatim}

Required packages include:
\begin{itemize}
    \item torch
    \item transformers
    \item scikit-learn
    \item pandas
    \item numpy
    \item nltk
    \item fastapi
    \item uvicorn
    \item beautifulsoup4
    \item requests
\end{itemize}

\section{Running the API}
\label{sec:running_api}

Start the FastAPI server:

\begin{verbatim}
cd api
uvicorn main:app --reload --port 8000
\end{verbatim}

The API will be available at \texttt{http://localhost:8000}.

Access the interactive API documentation at \texttt{http://localhost:8000/docs}.

\section{Running the Frontend}
\label{sec:running_frontend}

The frontend can be served using any static file server. For development:

\begin{verbatim}
cd frontend
python -m http.server 3000
\end{verbatim}

Access the frontend at \texttt{http://localhost:3000}.

\section{Training Models}
\label{sec:training}

\subsection{Download Training Data}
\label{subsec:download_data}

\begin{verbatim}
cd data
python download.py
\end{verbatim}

\subsection{Train Headline Model}
\label{subsec:train_headline}

Open and run all cells in \texttt{notebooks/train\_headline.ipynb}.

Training parameters:
\begin{itemize}
    \item Epochs: 5
    \item Batch size: 16
    \item Learning rate: 2e-5
    \item Max sequence length: 256
\end{itemize}

\subsection{Train Article Model}
\label{subsec:train_article}

Open and run all cells in \texttt{notebooks/train\_article.ipynb}.

Training parameters:
\begin{itemize}
    \item TF-IDF max features: 10,000
    \item n-gram range: (1, 2)
    \item Gradient Boosting estimators: 100
\end{itemize}

\section{Data Collection}
\label{sec:data_collection}

To collect new fact-checked statements from PolitiFact:

\begin{verbatim}
cd scraper
python politifact.py
\end{verbatim}

This will scrape statements and save them to a CSV file.

\section{Project Structure}
\label{sec:structure}

\begin{verbatim}
unfake/
├── api/
│   ├── main.py            # FastAPI application
│   ├── text_processing.py # Text preprocessing
│   └── types.py           # Pydantic models
├── data/
│   ├── download.py        # Data download script
│   ├── LIAR/              # LIAR dataset
│   ├── True_Fake/         # Article dataset
│   └── RoBERTa_Classifier/ # Trained model
├── frontend/
│   ├── index.html         # Main HTML page
│   ├── styles.css         # Stylesheet
│   └── app.js             # JavaScript logic
├── notebooks/
│   ├── train_headline.ipynb  # Headline model training
│   └── train_article.ipynb   # Article model training
├── scraper/
│   └── politifact.py      # PolitiFact scraper
└── report/                # LaTeX report files
\end{verbatim}

\section{Troubleshooting}
\label{sec:troubleshooting}

\subsection{CUDA Out of Memory}
\label{subsec:cuda_oom}

If you encounter GPU memory errors:
\begin{itemize}
    \item Reduce batch size in training notebooks
    \item Set \texttt{CUDA\_VISIBLE\_DEVICES=""} to use CPU
\end{itemize}

\subsection{Model Loading Errors}
\label{subsec:model_errors}

Ensure all model files are present in \texttt{data/RoBERTa\_Classifier/}:
\begin{itemize}
    \item model.safetensors
    \item config.json
    \item tokenizer.json
    \item vocab.json
    \item merges.txt
\end{itemize}

\subsection{CORS Errors}
\label{subsec:cors}

If the frontend cannot connect to the API, ensure CORS is properly configured in \texttt{api/main.py}.